\documentclass[a4paper, 10pt]{ctexart}
\usepackage[utf8]{inputenc}
\usepackage{geometry}[a4paper, 10pt]
\usepackage{listings}
\usepackage{color}
\usepackage{tcolorbox}
\usepackage{titlesec}
\usepackage{amsmath}

% 页面布局
\geometry{a4paper, left=2cm, right=2cm, top=2cm, bottom=2cm}

% 代码块设置
\definecolor{codegray}{gray}{0.9}
\definecolor{codeborder}{rgb}{0.8,0.8,0.8}
\tcbuselibrary{listingsutf8}
\newtcblisting{codeblock}[1][]{
  colback=codegray,
  colframe=codeborder,
  listing only,
  listing options={
    basicstyle=\ttfamily,
    breaklines=true,
    tabsize=2,
    showstringspaces=false,
    numbers=left,
    numberstyle=\tiny,
    xleftmargin=0.5em,
    xrightmargin=0.5em,
    aboveskip=1em,
    belowskip=1em,
    #1
  },
  enhanced,
  frame hidden,
  interior hidden,
  borderline west={2pt}{0pt}{codeborder},
}

% 章节标题设置
\titleformat{\section}{\Large\bfseries}{{第}\thesection{章}}{1em}{}
\titleformat{\subsection}{\large\bfseries}{\thesubsection}{1em}{}
\titleformat{\subsubsection}{\normalsize\bfseries}{\thesubsubsection}{1em}{}

\begin{document}

\title{Java程序设计精编教程(第3版)笔记}
\author{Liam}
\date{}

\maketitle

\section{Java入门}

\subsection{导读}
\begin{itemize}
  \item 主要内容:
  \begin{itemize}
    \item Java的平台无关性
    \item Java的地位
    \item 安装JDK
    \item 一个简单的Java应用程序
  \end{itemize}
\end{itemize}

\subsection{Java的平台无关性}
\begin{itemize}
  \item \textbf{Java虚拟机(Java Virtual Machine)}:
  \begin{itemize}
    \item Java可以在操作系统之上提供一个Java运行环境,由Java虚拟机、类库及核心文件组成。
    \item Java编译器不针对特定操作系统和CPU芯片进行编译,而是将Java源程序编译为字节码。
    \item \textbf{字节码}:一种中间代码,由Java虚拟机识别和执行。
    \item Java虚拟机将字节码翻译成当前平台的机器码并运行。
  \end{itemize}
  \item \textbf{平台无关性}:
  \begin{itemize}
    \item Java生成的字节码文件不依赖平台,可以在不同平台(如Windows、UNIX)上运行,只需对应的Java运行环境(JRE)。
  \end{itemize}
\end{itemize}

\subsection{Java之父 - James Gosling}
\begin{itemize}
  \item \textbf{历史背景}:
  \begin{itemize}
    \item 1990年,Sun公司成立了由James Gosling领导的开发小组,致力于开发可移植的跨平台语言。
    \item 1995年5月,Sun公司推出Java Development Kit(JDK) 1.0a2版本,标志着Java的诞生。
    \item 这个过程被形容为“有心栽花花不开,无心插柳柳成荫”。
  \end{itemize}
\end{itemize}

\subsection{Java的地位}
\begin{itemize}
  \item \textbf{网络地位}:
  \begin{itemize}
    \item Java的平台无关性使其成为编写网络应用程序的佼佼者,提供了许多网络应用核心技术。
  \end{itemize}
  \item \textbf{语言地位}:
  \begin{itemize}
    \item Java是面向对象编程语言,涉及网络、多线程等基础知识,是软件设计开发者应掌握的基础语言。
  \end{itemize}
  \item \textbf{需求地位}:
  \begin{itemize}
    \item IT行业对Java人才的需求不断增长,许多新技术领域都涉及Java语言。
  \end{itemize}
\end{itemize}

\subsection{安装JDK}
\begin{itemize}
  \item \textbf{Java平台版本}:
  \begin{itemize}
    \item Java SE(Java标准版或标准平台)
    \item Java EE(Java企业版或企业平台)
  \end{itemize}
  \item \textbf{虚拟机功能}:
  \begin{itemize}
    \item 虚拟机将字节码文件加载到内存并采用解释方式执行,根据平台指令翻译并执行字节码文件。
  \end{itemize}
  \item \textbf{JDK版本更新}:
  \begin{itemize}
    \item Oracle收购Sun公司后,Java更新频率从若干年一次变为每半年一次,大部分版本只有半年的支持期。
    \item 截至2023年上半年,最新版本为JDK20,实际生产环境中主要使用长期支持版本JDK8、11、17。
  \end{itemize}
  \item \textbf{JDK安装步骤}:
  \begin{itemize}
    \item 下载、安装JDK1.8,建议修改默认安装路径为:E:\textbackslash jdk1.8。
    \item 设置环境变量Path,添加JDK的bin目录路径。
  \end{itemize}
\end{itemize}

\subsection{Java程序的开发步骤}
\begin{enumerate}
  \item \textbf{编写源文件}:
  \begin{itemize}
    \item 扩展名为.java的源文件。
  \end{itemize}
  \item \textbf{编译源程序}:
  \begin{itemize}
    \item 使用Java编译器(javac.exe)编译源文件,生成字节码文件。
  \end{itemize}
  \item \textbf{运行程序}:
  \begin{itemize}
    \item 使用Java解释器(java.exe)解释执行字节码文件。
  \end{itemize}
\end{enumerate}

\subsection{一个简单的Java应用程序}
\begin{itemize}
  \item \textbf{例子1:Hello程序}:
  \begin{enumerate}
    \item \textbf{编写源文件}:
    \begin{codeblock}
public class Hello {
    public static void main (String args[]) {
        System.out.println("这是一个简单的Java应用程序");
    }
}
    \end{codeblock}
    \begin{itemize}
      \item 文件命名为Hello.java,保存至C:\textbackslash ch1。
      \item 注意良好的编码习惯和命名规则。
    \end{itemize}
    \item \textbf{编译源程序}:
    \begin{codeblock}
C:\ch1>javac Hello.java
    \end{codeblock}
    \item \textbf{运行程序}:
    \begin{codeblock}
C:\ch1>java Hello
    \end{codeblock}
  \end{enumerate}
\end{itemize}

\section{Java应用程序的基本结构}

\subsection{导读}
\begin{itemize}
  \item 主要内容:
  \begin{itemize}
    \item 问题的提出
    \item 简单的Circle类
    \item 使用Circle类创建对象
    \item 在Java应用程序中使用对象
    \item Java应用程序的基本结构
    \item 编程风格
  \end{itemize}
\end{itemize}

\subsection{问题的提出}
\begin{itemize}
  \item \textbf{计算圆面积的Java应用程序}:
  \begin{itemize}
    \item 如果多个应用程序都需要计算圆的面积,每个应用程序都需编写相同的代码。
    \item \textbf{目标}:将与圆有关的数据及计算代码封装,使得其他应用程序无需重复编写计算代码。
  \end{itemize}
\end{itemize}

\subsection{简单的Circle类}
\begin{itemize}
  \item \textbf{面向对象的抽象}:
  \begin{itemize}
    \item 抽象关键:数据和数据上的操作。
    \item \textbf{Circle类的定义}:
    \begin{codeblock}
class Circle {
    double radius;
    double getArea() {
        double area = 3.14 * radius * radius;
        return area;
    }
}
    \end{codeblock}
    \item 类声明和类体:\texttt{Circle.java}。
    \item 类体内容包括:
    \begin{itemize}
      \item \textbf{域变量(成员变量)}:描述圆的属性,如半径。
      \item \textbf{方法}:描述行为,如计算圆面积的方法。
    \end{itemize}
  \end{itemize}
\end{itemize}

\subsection{使用Circle类创建对象}
\begin{itemize}
  \item \textbf{类是Java中的数据类型}:
  \begin{itemize}
    \item 创建对象需经过两个步骤:声明对象和为对象分配变量。
  \end{itemize}
\end{itemize}

\subsubsection{用类声明对象}
\begin{itemize}
  \item \textbf{类声明变量}:
  \begin{itemize}
    \item 用类声明的变量即对象,例如:
    \begin{codeblock}
Circle circleOne;
    \end{codeblock}
    \item 声明后,\texttt{circleOne}是一个空对象,需为其分配变量。
  \end{itemize}
\end{itemize}

\subsubsection{为对象分配变量}
\begin{itemize}
  \item \textbf{分配变量}:
  \begin{itemize}
    \item 使用\texttt{new}运算符和构造方法,例如:
    \begin{codeblock}
circleOne = new Circle();
    \end{codeblock}
    \item 可以在声明时同时分配变量:
    \begin{codeblock}
Circle circleOne = new Circle();
    \end{codeblock}
  \end{itemize}
\end{itemize}

\subsubsection{使用对象}
\begin{itemize}
  \item \textbf{操作对象变量}:
  \begin{itemize}
    \item 使用\texttt{.}运算符操作对象的变量和调用方法,例如:
    \begin{codeblock}
circleOne.radius = 100;
double area = circleOne.getArea();
    \end{codeblock}
  \end{itemize}
\end{itemize}

\subsection{在应用程序中使用对象}
\begin{itemize}
  \item \textbf{例子:Circle类和Example2\_1类}:
  \begin{itemize}
    \item \textbf{Circle类}:
    \begin{codeblock}
class Circle {
    double radius;
    double getArea() {
        double area = 3.14 * radius * radius;
        return area;
    }
}
    \end{codeblock}
    \item \textbf{Example2\_1类}:
    \begin{codeblock}
public class Example2_1 {
    public static void main(String args[]) {
        Circle circleOne, circleTwo;
        circleOne = new Circle();
        circleTwo = new Circle();
        circleOne.radius = 123.86;
        circleTwo.radius = 69;
        double area = circleOne.getArea();
        System.out.println("circleOne的面积:" + area);
        area = circleTwo.getArea();
        System.out.println("circleTwo的面积:" + area);
    }
}
    \end{codeblock}
  \end{itemize}
\end{itemize}

\subsection{Java应用程序的基本结构}
\begin{itemize}
  \item \textbf{Java应用程序的组成}:
  \begin{itemize}
    \item 由若干个类构成,但必须有一个包含\texttt{main}方法的主类。
    \item 各类可存放在不同的源文件中或同一个源文件中。
  \end{itemize}
  \item \textbf{例子}:
  \begin{itemize}
    \item \texttt{Example2\_2.java}、\texttt{Rect.java}、\texttt{Lader.java}保存在\texttt{C:\textbackslash ch2}中,\texttt{Example2\_2.java}为主类源文件。
  \end{itemize}
\end{itemize}

\subsection{在一个源文件中编写多个类}
\begin{itemize}
  \item \textbf{编写多个类}:
  \begin{itemize}
    \item 一个源文件中可编写多个类,但只能有一个类使用\texttt{public}修饰。
    \item \textbf{例子3}:
    \begin{itemize}
      \item 命名保存源文件为\texttt{Rectangle.java}。
      \item 编译后会生成两个字节码文件。
      \item 运行时使用主类的名字。
    \end{itemize}
  \end{itemize}
\end{itemize}

\subsection{编程风格}
\begin{itemize}
  \item \textbf{Allman风格}:
  \begin{itemize}
    \item 也称“独行”风格,左、右大括号各自独占一行。
  \end{itemize}
  \item \textbf{Kernighan风格}:
  \begin{itemize}
    \item 也称“行尾”风格,左大括号在上一行的行尾,右大括号独占一行。
  \end{itemize}
  \item \textbf{注释}:
  \begin{itemize}
    \item 单行注释使用\texttt{//}。
    \item 多行注释使用\texttt{/* ... */}。
  \end{itemize}
\end{itemize}

\section{标识符与简单数据类型}

\subsection{导读}
\begin{itemize}
  \item 主要内容:
  \begin{itemize}
    \item 标识符与关键字
    \item 简单数据类型
    \item 简单数据类型的级别与类型转换
    \item 从命令行窗口输入、输出数据
  \end{itemize}
\end{itemize}

\subsection{标识符与关键字}
\begin{itemize}
  \item \textbf{标识符}:
  \begin{itemize}
    \item 用于标识类名、变量名、方法名、类型名、数组名、文件名的有效字符序列。
    \item 简单来说,标识符就是一个名字。
  \end{itemize}
  \item \textbf{关键字}:
  \begin{itemize}
    \item 关键字是Java语言中已经被赋予特定意义的单词。
    \item 关键字不能用作标识符。
  \end{itemize}
\end{itemize}

\subsection{简单数据类型}
\begin{itemize}
  \item \textbf{数据类型分类}:
  \begin{itemize}
    \item 基本数据类型:整数、小数、字符、布尔型
    \item 对象型数据:数组、由类生成的对象
  \end{itemize}
  \item \textbf{基本数据类型的解释}:
  \begin{itemize}
    \item 涉及字节数、编码格式、能参与的运算类型等。
  \end{itemize}
\end{itemize}

\subsubsection{逻辑类型}
\begin{itemize}
  \item \textbf{常量}:\texttt{true}, \texttt{false}
  \item \textbf{变量}:
  \begin{codeblock}
boolean xok = true;
boolean 关闭 = false;
  \end{codeblock}
\end{itemize}

\subsubsection{整数类型}
\begin{enumerate}
  \item \textbf{int 型}:
  \begin{itemize}
    \item 常量:123, 6000(十进制),077(八进制),0x3ABC(十六进制)
    \item 变量:
    \begin{codeblock}
int x = 12;
int 平均 = 9898;
    \end{codeblock}
    \item 占用4个字节(32位)
  \end{itemize}
  \item \textbf{byte 型}:
  \begin{itemize}
    \item 常量:无byte型常量表示法,可以用int型常量赋值给byte型变量
    \item 变量:
    \begin{codeblock}
byte x = -12;
byte tom = 28;
    \end{codeblock}
    \item 占用1个字节(8位)
  \end{itemize}
  \item \textbf{short 型}:
  \begin{itemize}
    \item 常量:无short型常量表示法,可以用int型常量赋值给short型变量
    \item 变量:
    \begin{codeblock}
short x = 12;
short y = 1234;
    \end{codeblock}
    \item 占用2个字节(16位)
  \end{itemize}
  \item \textbf{long 型}:
  \begin{itemize}
    \item 常量:用后缀L表示
    \begin{codeblock}
long width = 12L;
long height = 2005L;
    \end{codeblock}
    \item 占用8个字节(64位)
  \end{itemize}
\end{enumerate}

\subsubsection{字符类型}
\begin{itemize}
  \item \textbf{常量}:
  \begin{itemize}
    \item 用单引号扩起的Unicode表中的一个字符,例如:'A', 'b', '好', '\textbackslash t'
  \end{itemize}
  \item \textbf{变量}:
  \begin{codeblock}
char ch = 'A';
char home = '家';
  \end{codeblock}
  \item \textbf{转义字符}:
  \begin{itemize}
    \item 例如:\texttt{\textbackslash n}(换行),\texttt{\textbackslash b}(退格),\texttt{\textbackslash t}(水平制表),\texttt{\textbackslash '}(单引号),\texttt{\textbackslash "}(双引号),\texttt{\textbackslash \textbackslash}(反斜线)
  \end{itemize}
\end{itemize}

\subsubsection{浮点类型}
\begin{enumerate}
  \item \textbf{float 型}:
  \begin{itemize}
    \item 常量:453.5439f, 2e40f
    \item 变量:
    \begin{codeblock}
float x = 22.76f;
float tom = 1234.987f;
    \end{codeblock}
    \item 占用4个字节(32位),精度保留8位有效数字
  \end{itemize}
  \item \textbf{double 型}:
  \begin{itemize}
    \item 常量:2389.539d, 1e-90
    \item 变量:
    \begin{codeblock}
double height = 23.345;
double width = 34.56D;
    \end{codeblock}
    \item 占用8个字节(64位),精度保留16位有效数字
  \end{itemize}
\end{enumerate}

\subsection{简单数据类型的级别与类型转换运算}
\begin{itemize}
  \item \textbf{数据类型精度排列}(不包括逻辑类型):
  \begin{itemize}
    \item byte < short < char < int < long < float < double
  \end{itemize}
  \item \textbf{自动类型转换}:
  \begin{itemize}
    \item 低级别变量赋值给高级别变量时,系统自动转换,例如:
    \begin{codeblock}
float x = 100;
    \end{codeblock}
  \end{itemize}
  \item \textbf{强制类型转换}:
  \begin{itemize}
    \item 高级别变量赋值给低级别变量时,必须使用强制类型转换,例如:
    \begin{codeblock}
int x = (int) 34.89;
    \end{codeblock}
  \end{itemize}
  \item \textbf{范围超出处理}:
  \begin{itemize}
    \item int型常量超出byte和short型变量范围时,需进行类型转换,例如:
    \begin{codeblock}
byte a = (byte) 128;  // 超出范围会导致精度损失
    \end{codeblock}
  \end{itemize}
\end{itemize}

\subsection{从命令行输入、输出数据}
\subsubsection{输入基本型数据}
\begin{itemize}
  \item \textbf{使用Scanner类}:
  \begin{codeblock}
Scanner reader = new Scanner(System.in);
  \end{codeblock}
  \item \textbf{读取用户输入}:
  \begin{itemize}
    \item \texttt{nextBoolean()}, \texttt{nextByte()}, \texttt{nextShort()}, \texttt{nextInt()}, \texttt{nextLong()}, \texttt{nextFloat()}, \texttt{nextDouble()}
  \end{itemize}
\end{itemize}

\subsubsection{输出基本型数据}
\begin{itemize}
  \item \textbf{使用System.out}:
  \begin{itemize}
    \item \texttt{System.out.println()}: 输出后换行
    \item \texttt{System.out.print()}: 输出后不换行
  \end{itemize}
  \item \textbf{并置符号}:
  \begin{itemize}
    \item 使用\texttt{+}将变量、表达式或常数值与字符串并置输出,例如:
    \begin{codeblock}
System.out.println(m + "个数的和为" + sum);
    \end{codeblock}
  \end{itemize}
  \item \textbf{格式化输出}:
  \begin{itemize}
    \item JDK1.5新增\texttt{printf}方法,类似C语言中的\texttt{printf},例如:
    \begin{codeblock}
System.out.printf("%d %f %s", 123, 456.789, "Hello");
    \end{codeblock}
  \end{itemize}
\end{itemize}

\section{运算符、表达式与语句}

\subsection{导读}
\begin{itemize}
  \item 主要内容:
  \begin{itemize}
    \item 运算符与表达式
    \item 语句概述
    \item if条件分支语句
    \item switch开关语句
    \item 循环语句
    \item break和continue语句
    \item 数组
  \end{itemize}
\end{itemize}

\subsection{运算符与表达式}
\begin{itemize}
  \item \textbf{Java提供了丰富的运算符}:
  \begin{itemize}
    \item 算术运算符、关系运算符、逻辑运算符、位运算符等。
  \end{itemize}
\end{itemize}

\subsubsection{算术运算符与算术表达式}
\begin{enumerate}
  \item \textbf{加减运算符}:\texttt{+},\texttt{-}
  \begin{itemize}
    \item 二目运算符,结合方向从左到右,操作元为整型或浮点型数据,优先级为4级。
  \end{itemize}
  \item \textbf{乘、除和求余运算符}:\texttt{*},\texttt{/},\texttt{\%}
  \begin{itemize}
    \item 二目运算符,结合方向从左到右,操作元为整型或浮点型数据,优先级为3级。
  \end{itemize}
  \item \textbf{算术表达式}:
  \begin{itemize}
    \item 用算术符号和括号连接起来的符合Java语法规则的式子。
  \end{itemize}
\end{enumerate}

\subsubsection{自增,自减运算符}
\begin{itemize}
  \item \textbf{自增、自减运算符}:\texttt{++},\texttt{--}
  \begin{itemize}
    \item 单目运算符,可以放在操作元之前或之后,操作元必须是整型或浮点型变量。
    \item \texttt{++x}表示在使用x之前,先使x的值增1。
    \item \texttt{x++}表示在使用x之后,使x的值增1。
  \end{itemize}
\end{itemize}

\subsubsection{算术混合运算的精度}
\begin{itemize}
  \item \textbf{精度从低到高排列顺序}:byte, short, char, int, long, float, double
  \item \textbf{计算规则}:
  \begin{enumerate}
    \item 有double型数据时,按double精度运算。
    \item 有float型数据时,按float精度运算。
    \item 有long型数据时,按long精度运算。
    \item 精度低于int型整数时,按int精度运算。
  \end{enumerate}
\end{itemize}

\subsubsection{关系运算符与关系表达式}
\begin{itemize}
  \item \textbf{关系运算符}:
  \begin{itemize}
    \item 二目运算符,用来比较两个值的关系,结果为boolean型,成立时结果为true,否则为false。
  \end{itemize}
\end{itemize}

\subsubsection{逻辑运算符与逻辑表达式}
\begin{itemize}
  \item \textbf{逻辑运算符}:\texttt{\&\&},\texttt{||},\texttt{!}
  \begin{itemize}
    \item \texttt{\&\&}和\texttt{||}为二目运算符,实现逻辑与、逻辑或。
    \item \texttt{!}为单目运算符,实现逻辑非。
    \item 操作元必须是boolean型数据,逻辑运算符可连接关系表达式。
  \end{itemize}
\end{itemize}

\subsubsection{赋值运算符与赋值表达式}
\begin{itemize}
  \item \textbf{赋值运算符}:\texttt{=}
  \begin{itemize}
    \item 二目运算符,左操作元必须是变量,优先级较低(14级),结合方向从右到左。
  \end{itemize}
\end{itemize}

\subsubsection{位运算符}
\begin{itemize}
  \item \textbf{位运算符}:
  \begin{enumerate}
    \item “按位与”运算:\texttt{\&}
    \item “按位或”运算:\texttt{|}
    \item “按位非”运算:\texttt{\~{}}
    \item “按位异或”运算:\texttt{\^{}}
  \end{enumerate}
  \item 参与运算的为整型数据,结果也是整型数据。
\end{itemize}

\subsubsection{instanceof 运算符}
\begin{itemize}
  \item \textbf{instanceof运算符}:
  \begin{itemize}
    \item 二目运算符,左操作元是对象,右操作元是类,当对象是该类或其子类创建时,结果为true,否则为false。
  \end{itemize}
\end{itemize}

\subsubsection{运算符综述}
\begin{itemize}
  \item \textbf{表达式}:
  \begin{itemize}
    \item 用运算符连接的符合Java规则的式子。
    \item 运算符的优先级和结合性决定表达式中运算的先后顺序。
    \item 推荐使用括号来明确运算次序。
  \end{itemize}
\end{itemize}

\subsection{语句概述}
\begin{itemize}
  \item \textbf{Java语句的分类}:
  \begin{enumerate}
    \item \textbf{方法调用语句}:如\texttt{System.out.println("Hello");}
    \item \textbf{表达式语句}:如\texttt{x=23;}
    \item \textbf{复合语句}:用\texttt{\{\}}括起来的语句块
    \item \textbf{空语句}:单个分号
    \item \textbf{控制语句}:条件分支、开关语句、循环语句
    \item \textbf{package语句和import语句}:将在第5章讲解
  \end{enumerate}
\end{itemize}

\subsection{if条件分支语句}
\begin{itemize}
  \item \textbf{if语句}:
  \begin{itemize}
    \item 单条件分支语句,根据一个条件控制程序执行流程。
  \end{itemize}
\end{itemize}

\subsubsection{if语句}
\begin{itemize}
  \item \textbf{语法格式}:
  \begin{codeblock}
if(表达式) { 
    若干语句
}
  \end{codeblock}
\end{itemize}

\subsubsection{if-else 语句}
\begin{itemize}
  \item \textbf{语法格式}:
  \begin{codeblock}
if(表达式) {
    若干语句
} else {
    若干语句
}
  \end{codeblock}
\end{itemize}

\subsubsection{if-else if-else 语句}
\begin{itemize}
  \item \textbf{语法格式}:
  \begin{codeblock}
if(表达式) {
    若干语句
} else if(表达式) {
    若干语句
} else {
    若干语句
}
  \end{codeblock}
\end{itemize}

\subsection{switch开关语句}
\begin{itemize}
  \item \textbf{switch语句}:
  \begin{itemize}
    \item 单条件多分支语句,语法格式如下:
    \begin{codeblock}
switch(表达式) {
    case 常量值1:
        若干个语句
        break;
    case 常量值2:
        若干个语句
        break;
    // ...
    default:
        若干语句
}
    \end{codeblock}
  \end{itemize}
\end{itemize}

\subsection{循环语句}
\begin{itemize}
  \item \textbf{循环语句}:
  \begin{itemize}
    \item 包括for循环、while循环和do-while循环。
  \end{itemize}
\end{itemize}

\subsubsection{for循环语句}
\begin{itemize}
  \item \textbf{语法格式}:
  \begin{codeblock}
for(表达式1; 表达式2; 表达式3) {
    若干语句
}
  \end{codeblock}
  \item \textbf{执行规则}:
  \begin{enumerate}
    \item 计算表达式1,完成初始化。
    \item 判断表达式2,若为true,则执行循环体,再计算表达式3。
    \item 若表达式2为false,则结束for语句。
  \end{enumerate}
\end{itemize}

\subsubsection{while 循环}
\begin{itemize}
  \item \textbf{语法格式}:
  \begin{codeblock}
while(表达式) {
    若干语句
}
  \end{codeblock}
  \item \textbf{执行规则}:
  \begin{enumerate}
    \item 计算表达式的值,若为true,则执行循环体,再计算表达式。
    \item 若为false,则结束while语句。
  \end{enumerate}
\end{itemize}

\subsubsection{do-while循环}
\begin{itemize}
  \item \textbf{语法格式}:
  \begin{codeblock}
do {
    若干语句
} while(表达式);
  \end{codeblock}
  \item \textbf{执行规则}:
  \begin{enumerate}
    \item 执行循环体,再计算表达式的值。
    \item 若为true,则继续执行循环体,否则结束do-while语句。
  \end{enumerate}
\end{itemize}

\subsection{break和continue语句}
\begin{itemize}
  \item \textbf{break语句}:
  \begin{itemize}
    \item 用于结束整个循环语句。
  \end{itemize}
  \item \textbf{continue语句}:
  \begin{itemize}
    \item 用于结束本次循环,进入下一次循环。
  \end{itemize}
\end{itemize}

\subsection{数组}
\begin{itemize}
  \item \textbf{数组}:
  \begin{itemize}
    \item 相同类型变量按顺序组成的复合数据类型,称为数组的元素或单元。
    \item 数组通过数组名加索引使用元素,创建数组需声明和分配变量两个步骤。
  \end{itemize}
\end{itemize}

\subsubsection{声明数组}
\begin{itemize}
  \item \textbf{一维数组声明格式}:
  \begin{codeblock}
元素类型 数组名[];
元素类型[] 数组名;
  \end{codeblock}
  \item \textbf{二维数组声明格式}:
  \begin{codeblock}
元素类型 数组名[][];
元素类型[][] 数组名;
  \end{codeblock}
\end{itemize}

\subsubsection{为数组分配元素}
\begin{itemize}
  \item \textbf{分配内存空间格式}:
  \begin{codeblock}
数组名 = new 元素类型[元素个数];
  \end{codeblock}
\end{itemize}

\subsubsection{数组元素的使用}
\begin{itemize}
  \item \textbf{一维数组}:
  \begin{itemize}
    \item 通过索引符访问元素,如\texttt{boy[0]}。
  \end{itemize}
  \item \textbf{二维数组}:
  \begin{itemize}
    \item 通过索引符访问元素,如\texttt{a[0][1]}。
  \end{itemize}
\end{itemize}

\subsubsection{length的使用}
\begin{itemize}
  \item \textbf{数组长度}:
  \begin{itemize}
    \item 一维数组:\texttt{数组名.length}
    \item 二维数组:\texttt{数组名.length}为包含的一维数组个数。
  \end{itemize}
\end{itemize}

\subsubsection{数组的初始化}
\begin{itemize}
  \item \textbf{声明时初始化}:
  \begin{codeblock}
float boy[] = {21.3f, 23.89f, 2.0f, 23f, 778.98f};
  \end{codeblock}
  \item \textbf{二维数组初始化}:
  \begin{codeblock}
int a[][] = {{1}, {11}, {121}, {1331}, {14641}};
  \end{codeblock}
\end{itemize}

\subsubsection{数组的引用}
\begin{itemize}
  \item \textbf{数组属于引用型变量}:
  \begin{itemize}
    \item 两个相同类型的数组具有相同引用时,它们有完全相同的元素。
    \begin{codeblock}
int a[] = {1, 2, 3};
int b[];
b = a;
    \end{codeblock}
  \end{itemize}
\end{itemize}

\subsubsection{遍历数组}
\begin{enumerate}
  \item \textbf{基于循环语句的遍历}:
  \begin{codeblock}
for(声明循环变量 : 数组名) {
    // 语句
}
  \end{codeblock}
  \item \textbf{使用toString()方法遍历数组}:
  \begin{codeblock}
int[] a = {1, 2, 3, 4, 5, 6};
System.out.println(Arrays.toString(a));
  \end{codeblock}
\end{enumerate}

\section{类与对象}

\subsection{导读}
\begin{itemize}
  \item 主要内容:
  \begin{itemize}
    \item 面向对象的特性
    \item 类
    \item 构造方法与对象的创建
    \item 参数传值
    \item 对象的组合
    \item 实例成员与类成员
    \item 方法重载与多态
    \item this关键字
    \item 包
    \item import语句,访问权限
    \item 可变参数
  \end{itemize}
\end{itemize}

\subsection{面向对象的特性}
\begin{itemize}
  \item \textbf{面向对象编程的四个特性}:
  \begin{itemize}
    \item 抽象
    \item 封装
    \item 继承
    \item 多态
  \end{itemize}
\end{itemize}

\subsection{类}
\begin{itemize}
  \item \textbf{类的定义}:
  \begin{itemize}
    \item 类是Java程序的基本要素,封装对象的状态和方法,是定义对象的模板。
    \item 类的实现包括类声明和类体,格式如下:
    \begin{codeblock}
class 类名 {
    类体的内容
}
    \end{codeblock}
  \end{itemize}
\end{itemize}

\subsubsection{类声明}
\begin{itemize}
  \item \textbf{类声明格式}:
  \begin{itemize}
    \item \texttt{class 类名}
    \item 类名首字母大写,容易识别,见名知意。
  \end{itemize}
\end{itemize}

\subsubsection{类体}
\begin{itemize}
  \item \textbf{类体内容}:
  \begin{itemize}
    \item 由变量定义(刻画属性)和方法定义(刻画功能)构成。
  \end{itemize}
\end{itemize}

\subsubsection{成员变量和局部变量}
\begin{itemize}
  \item \textbf{成员变量}:
  \begin{itemize}
    \item 在类体中声明,整个类内有效。
  \end{itemize}
  \item \textbf{局部变量}:
  \begin{itemize}
    \item 在方法体中声明,仅在方法内有效。
  \end{itemize}
  \item \textbf{隐藏规则}:
  \begin{itemize}
    \item 局部变量名与成员变量名相同时,成员变量在方法内暂时失效。
  \end{itemize}
\end{itemize}

\subsubsection{方法}
\begin{itemize}
  \item \textbf{方法定义}:
  \begin{itemize}
    \item 包括方法声明和方法体。
    \item 方法声明部分包含方法名和返回类型,格式如下:
    \begin{codeblock}
返回类型 方法名() {
    方法体的内容
}
    \end{codeblock}
  \end{itemize}
\end{itemize}

\subsubsection{注意事项}
\begin{itemize}
  \item \textbf{成员变量操作}:
  \begin{itemize}
    \item 只能在方法中操作成员变量,不可在类体中单独赋值。
  \end{itemize}
\end{itemize}

\subsubsection{类的UML图}
\begin{itemize}
  \item \textbf{UML图}:
  \begin{itemize}
    \item 用于描述系统静态结构,包括类图、接口图、泛化关系、关联关系、依赖关系、实现关系。
    \item 类图包括名字层、变量层(属性层)、方法层(操作层)。
  \end{itemize}
\end{itemize}

\subsection{构造方法与对象的创建}
\begin{itemize}
  \item \textbf{对象创建}:
  \begin{itemize}
    \item 类声明的变量称为对象,需通过构造方法创建对象,格式如下:
    \begin{codeblock}
类名 对象名 = new 类名();
    \end{codeblock}
  \end{itemize}
\end{itemize}

\subsubsection{构造方法}
\begin{itemize}
  \item \textbf{构造方法}:
  \begin{itemize}
    \item 特殊方法,名字与类名相同,无返回类型。
    \item 一个类可有多个构造方法,但参数必须不同。
  \end{itemize}
\end{itemize}

\subsubsection{创建对象}
\begin{itemize}
  \item \textbf{步骤}:
  \begin{enumerate}
    \item 声明对象:\texttt{Lader lader;}
    \item 分配内存:\texttt{lader = new Lader();}
  \end{enumerate}
\end{itemize}

\subsubsection{使用对象}
\begin{itemize}
  \item \textbf{操作对象}:
  \begin{itemize}
    \item 使用\texttt{.}操作变量和调用方法。
  \end{itemize}
\end{itemize}

\subsubsection{对象的引用和实体}
\begin{itemize}
  \item \textbf{对象的引用}:
  \begin{itemize}
    \item 对象存放着引用,引用相同的对象具有相同的实体。
  \end{itemize}
\end{itemize}

\subsection{参数传值}
\begin{itemize}
  \item \textbf{参数传值}:
  \begin{itemize}
    \item 方法的参数为局部变量,调用方法时,参数变量必须有具体值。
  \end{itemize}
\end{itemize}

\subsubsection{传值机制}
\begin{itemize}
  \item \textbf{传值}:
  \begin{itemize}
    \item 所有参数为传值,传递的是值的拷贝。
  \end{itemize}
\end{itemize}

\subsubsection{基本数据类型参数的传值}
\begin{itemize}
  \item \textbf{基本数据类型传值}:
  \begin{itemize}
    \item 传递的值级别不高于参数级别。
  \end{itemize}
\end{itemize}

\subsubsection{引用类型参数的传值}
\begin{itemize}
  \item \textbf{引用类型传值}:
  \begin{itemize}
    \item 传递的是变量中的引用,而不是实体。
  \end{itemize}
\end{itemize}

\subsection{对象的组合}
\begin{itemize}
  \item \textbf{对象组合}:
  \begin{itemize}
    \item 一个类可将对象作为成员变量,创建的对象包含其他对象。
  \end{itemize}
\end{itemize}

\subsection{实例成员与类成员}
\begin{itemize}
  \item \textbf{实例变量和类变量}:
  \begin{itemize}
    \item 实例变量不加\texttt{static}修饰,类变量加\texttt{static}修饰。
  \end{itemize}
\end{itemize}

\subsubsection{实例变量和类变量的声明}
\begin{itemize}
  \item \textbf{声明格式}:
  \begin{itemize}
    \item 实例变量不加\texttt{static},类变量加\texttt{static}。
  \end{itemize}
\end{itemize}

\subsubsection{实例变量和类变量的区别}
\begin{enumerate}
  \item 不同对象的实例变量互不相同。
  \item 所有对象共享类变量。
  \item 类变量可通过类名直接访问。
\end{enumerate}

\subsubsection{实例方法和类方法的定义}
\begin{itemize}
  \item \textbf{方法定义}:
  \begin{itemize}
    \item 实例方法不加\texttt{static},类方法加\texttt{static}。
  \end{itemize}
\end{itemize}

\subsubsection{实例方法和类方法的区别}
\begin{enumerate}
  \item \textbf{对象调用实例方法}:
  \begin{itemize}
    \item 实例方法操作实例变量。
  \end{itemize}
  \item \textbf{类名调用类方法}:
  \begin{itemize}
    \item 类方法不操作实例变量。
  \end{itemize}
\end{enumerate}

\subsection{方法重载与多态}
\begin{itemize}
  \item \textbf{方法重载}:
  \begin{itemize}
    \item 一个类中多个方法具有相同名字,但参数不同。
  \end{itemize}
\end{itemize}

\subsection{this关键字}
\begin{itemize}
  \item \textbf{this关键字}:
  \begin{itemize}
    \item 表示某个对象,可在实例方法和构造方法中使用。
  \end{itemize}
\end{itemize}

\subsubsection{在构造方法中使用this}
\begin{itemize}
  \item \textbf{使用方式}:
  \begin{itemize}
    \item 表示使用该构造方法创建的对象。
  \end{itemize}
\end{itemize}

\subsubsection{在实例方法中使用this}
\begin{itemize}
  \item \textbf{使用方式}:
  \begin{itemize}
    \item 表示调用该方法的当前对象。
  \end{itemize}
\end{itemize}

\subsection{包}
\begin{itemize}
  \item \textbf{包机制}:
  \begin{itemize}
    \item 用于管理类,区分名字相同的类。
  \end{itemize}
\end{itemize}

\subsubsection{包语句}
\begin{itemize}
  \item \textbf{声明格式}:
  \begin{codeblock}
package 包名;
  \end{codeblock}
\end{itemize}

\subsubsection{有包名的类的存储目录}
\begin{itemize}
  \item \textbf{目录结构}:
  \begin{itemize}
    \item 存储文件的目录结构需包含包名结构。
  \end{itemize}
\end{itemize}

\subsubsection{运行有包名的主类}
\begin{itemize}
  \item \textbf{运行格式}:
  \begin{codeblock}
java 包名.主类名
  \end{codeblock}
\end{itemize}

\subsection{import 语句}
\begin{itemize}
  \item \textbf{引入类}:
  \begin{itemize}
    \item 使用\texttt{import}语句引入不同包中的类。
  \end{itemize}
\end{itemize}

\subsubsection{引入类库中的类}
\begin{itemize}
  \item \textbf{引入格式}:
  \begin{codeblock}
import java.util.Date;
  \end{codeblock}
\end{itemize}

\subsubsection{引入自定义包中的类}
\begin{itemize}
  \item \textbf{引入格式}:
  \begin{codeblock}
import 包名.*;
  \end{codeblock}
\end{itemize}

\subsection{访问权限}
\begin{itemize}
  \item \textbf{访问权限}:
  \begin{itemize}
    \item 对象是否可以操作自己的变量或使用类中的方法。
  \end{itemize}
\end{itemize}

\subsubsection{何谓访问权限}
\begin{itemize}
  \item \textbf{访问修饰符}:
  \begin{itemize}
    \item \texttt{private}、\texttt{protected}、\texttt{public}用于修饰成员变量或方法。
  \end{itemize}
\end{itemize}

\subsubsection{私有变量和私有方法}
\begin{itemize}
  \item \textbf{私有修饰符}:
  \begin{itemize}
    \item \texttt{private}修饰的成员变量和方法,仅在本类中访问。
  \end{itemize}
\end{itemize}

\subsubsection{共有变量和共有方法}
\begin{itemize}
  \item \textbf{公共修饰符}:
  \begin{itemize}
    \item \texttt{public}修饰的成员变量和方法,可在任何类中访问。
  \end{itemize}
\end{itemize}

\subsubsection{友好变量和友好方法}
\begin{itemize}
  \item \textbf{友好访问}:
  \begin{itemize}
    \item 在同一包中的类中访问。
  \end{itemize}
\end{itemize}

\subsubsection{受保护的成员变量和方法}
\begin{itemize}
  \item \textbf{保护修饰符}:
  \begin{itemize}
    \item \texttt{protected}修饰的成员变量和方法,可在同一包和子类中访问。
  \end{itemize}
\end{itemize}

\subsubsection{public类与友好类}
\begin{itemize}
  \item \textbf{类修饰符}:
  \begin{itemize}
    \item \texttt{public}类在任何类中使用,不加\texttt{public}为友好类,仅在同一包中使用。
  \end{itemize}
\end{itemize}

\subsection{基本类型的类包装}
\begin{itemize}
  \item \textbf{类包装}:
  \begin{itemize}
    \item Java提供类包装基本数据类型,如\texttt{Byte}、\texttt{Integer}、\texttt{Short}、\texttt{Long}、\texttt{Float}、\texttt{Double}和\texttt{Character}。
  \end{itemize}
\end{itemize}

\subsubsection{Double和Float类}
\begin{itemize}
  \item \textbf{类包装}:
  \begin{itemize}
    \item \texttt{Double}和\texttt{Float}类包装\texttt{double}和\texttt{float}类型,提供相应的方法获取基本数据类型值。
  \end{itemize}
\end{itemize}

\subsubsection{Byte、Short、Integer、Long类}
\begin{itemize}
  \item \textbf{类包装}:
  \begin{itemize}
    \item \texttt{Byte}、\texttt{Short}、\texttt{Integer}、\texttt{Long}类包装对应基本数据类型,提供相应方法获取值。
  \end{itemize}
\end{itemize}

\subsubsection{Character类}
\begin{itemize}
  \item \textbf{类包装}:
  \begin{itemize}
    \item \texttt{Character}类包装\texttt{char}类型,提供方法判断字符属性和转换字符大小写。
  \end{itemize}
\end{itemize}

\subsection{可变参数}
\begin{itemize}
  \item \textbf{可变参数}:
  \begin{itemize}
    \item 方法参数列表中从某项至最后一项参数数量不固定,格式为\texttt{类型... 参数名}。
  \end{itemize}
\end{itemize}

\section{子类与继承}

\subsection{导读}
\begin{itemize}
  \item 主要内容:
  \begin{itemize}
    \item 子类与超类
    \item 使用super关键字
    \item final修饰符
    \item 向上转型
    \item 继承层次
    \item 继承的优点和缺点
  \end{itemize}
\end{itemize}

\subsection{子类与超类}
\begin{itemize}
  \item \textbf{继承}:
  \begin{itemize}
    \item 通过扩展已有的类,创建新类。新类称为子类或派生类,已有的类称为超类或基类。
    \item 继承允许子类继承超类的属性和方法。
  \end{itemize}
\end{itemize}

\subsubsection{继承的实现}
\begin{itemize}
  \item \textbf{定义子类}:
  \begin{itemize}
    \item 使用\texttt{extends}关键字定义子类,格式如下:
    \begin{codeblock}
class 子类 extends 超类 {
    // 子类的特有成员变量
    // 子类的特有方法
}
    \end{codeblock}
  \end{itemize}
\end{itemize}

\subsubsection{继承的特点}
\begin{itemize}
  \item 子类拥有超类的所有成员变量和方法,但不包括超类的构造方法。
  \item 子类可以添加新的成员变量和方法,也可以重写超类的方法。
\end{itemize}

\subsection{使用super关键字}
\begin{itemize}
  \item \textbf{super关键字}:
  \begin{itemize}
    \item 用于调用超类的构造方法和超类的成员变量和方法。
  \end{itemize}
\end{itemize}

\subsubsection{调用超类构造方法}
\begin{itemize}
  \item \textbf{格式}:
  \begin{codeblock}
super(参数列表);
  \end{codeblock}
  \item \textbf{注意}:
  \begin{itemize}
    \item 必须在子类构造方法的第一行调用超类的构造方法。
  \end{itemize}
\end{itemize}

\subsubsection{调用超类成员变量和方法}
\begin{itemize}
  \item \textbf{格式}:
  \begin{codeblock}
super.成员变量;
super.方法名(参数列表);
  \end{codeblock}
\end{itemize}

\subsection{final修饰符}
\begin{itemize}
  \item \textbf{final类}:
  \begin{itemize}
    \item 不能被继承的类,用\texttt{final}关键字修饰。
  \end{itemize}
  \item \textbf{final方法}:
  \begin{itemize}
    \item 不能被子类重写的方法,用\texttt{final}关键字修饰。
  \end{itemize}
  \item \textbf{final变量}:
  \begin{itemize}
    \item 值不能改变的变量,用\texttt{final}关键字修饰。
  \end{itemize}
\end{itemize}

\subsection{向上转型}
\begin{itemize}
  \item \textbf{向上转型}:
  \begin{itemize}
    \item 将子类对象的引用赋给超类变量。
    \item 格式:
    \begin{codeblock}
超类 对象名 = new 子类();
    \end{codeblock}
  \end{itemize}
  \item \textbf{特点}:
  \begin{itemize}
    \item 只能访问超类中的成员变量和方法,不能访问子类的特有成员变量和方法。
  \end{itemize}
\end{itemize}

\subsection{继承层次}
\begin{itemize}
  \item \textbf{继承层次}:
  \begin{itemize}
    \item 多层继承形成继承层次结构,最顶层的类称为根类。
    \item Java中所有类最终继承自\texttt{Object}类。
  \end{itemize}
\end{itemize}

\subsection{继承的优点和缺点}
\begin{itemize}
  \item \textbf{优点}:
  \begin{itemize}
    \item 提高代码的复用性。
    \item 提高代码的可维护性。
    \item 提供多态性。
  \end{itemize}
  \item \textbf{缺点}:
  \begin{itemize}
    \item 增加了类之间的耦合性。
    \item 继承层次过深,会使系统复杂化。
  \end{itemize}
\end{itemize}

\section{接口与实现}

\subsection{导读}
\begin{itemize}
  \item 主要内容:
  \begin{itemize}
    \item 接口
    \item 实现接口
    \item 理解接口
    \item 接口回调
    \item 接口与多态
    \item 接口变量做参数
    \item 面向接口编程
  \end{itemize}
\end{itemize}

\subsection{接口}
\begin{itemize}
  \item \textbf{接口概念}:
  \begin{itemize}
    \item 接口克服了Java单继承的限制,一个类可以实现多个接口。
    \item 使用\texttt{interface}关键字定义接口,格式如下:
    \begin{codeblock}
interface 接口名 {
    final int MAX = 100;
    void add();
    float sum(float x, float y);
}
    \end{codeblock}
    \item 接口包含常量定义和方法定义。
  \end{itemize}
\end{itemize}

\subsection{实现接口}
\begin{itemize}
  \item \textbf{实现接口}:
  \begin{itemize}
    \item 一个类通过\texttt{implements}关键字声明实现一个或多个接口,例如:
    \begin{codeblock}
class A implements Printable, Addable {
    ...
}
    \end{codeblock}
    \item 类实现接口时,必须重写接口的所有方法。
    \item Java提供的接口在相应包中,通过\texttt{import}语句引入包中的类和接口,例如:
    \begin{codeblock}
import java.io.*;
    \end{codeblock}
  \end{itemize}
\end{itemize}

\subsection{理解接口}
\begin{itemize}
  \item \textbf{接口功能}:
  \begin{itemize}
    \item 接口增加许多类需要具有的功能,不同类可以实现相同接口,同一类也可以实现多个接口。
    \item 接口只关心操作,不关心具体实现。
  \end{itemize}
\end{itemize}

\subsection{接口的UML图}
\begin{itemize}
  \item \textbf{接口UML图}:
  \begin{itemize}
    \item 类似类的UML图,使用一个长方形描述接口的主要构成,分为名字层、常量层、方法层。
  \end{itemize}
\end{itemize}

\subsection{接口回调}
\begin{itemize}
  \item \textbf{接口回调}:
  \begin{itemize}
    \item 可以将实现接口的类创建的对象引用赋给接口变量,接口变量可以调用被类重写的方法。
    \item 当接口变量调用重写的方法时,通知相应对象调用该方法。
  \end{itemize}
\end{itemize}

\subsection{接口与多态}
\begin{itemize}
  \item \textbf{接口与多态}:
  \begin{itemize}
    \item 接口中声明若干个抽象方法,具体实现由实现接口的类完成。
    \item 接口回调的核心思想是接口变量存放实现接口类的对象引用,从而回调类实现的方法。
  \end{itemize}
\end{itemize}

\subsection{接口变量做参数}
\begin{itemize}
  \item \textbf{接口参数}:
  \begin{itemize}
    \item 方法的参数是接口类型时,可以传递实现该接口的类实例的引用,接口参数可以回调类实现的方法。
  \end{itemize}
\end{itemize}

\subsection{abstract类与接口的比较}
\begin{itemize}
  \item \textbf{比较}:
  \begin{enumerate}
    \item 抽象类和接口都可以有抽象方法。
    \item 接口中只可以有常量,不能有变量;抽象类中即可以有常量也可以有变量。
    \item 抽象类中可以有非抽象方法,接口不可以。
  \end{enumerate}
\end{itemize}

\subsection{面向接口编程}
\begin{itemize}
  \item \textbf{面向接口编程}:
  \begin{itemize}
    \item 在接口中声明若干个抽象方法,方法实现由实现接口的类完成。
    \item 核心思想是使用接口回调,接口变量存放实现接口类的对象引用,从而回调类实现的方法。
  \end{itemize}
\end{itemize}

\section{内部类与异常类}

\subsection{导读}
\begin{itemize}
  \item 主要内容:
  \begin{itemize}
    \item 内部类
    \item 匿名类
    \item 异常类
    \item 断言
  \end{itemize}
\end{itemize}

\subsection{内部类}
\begin{itemize}
  \item \textbf{内部类定义}:
  \begin{itemize}
    \item Java支持在一个类中声明另一个类,这样的类称为内部类,包含内部类的类称为外嵌类。
    \item 内部类的类体中不可以声明类变量和类方法。
    \item 外嵌类的类体中可以用内部类声明对象作为外嵌类的成员。
    \item 内部类仅供它的外嵌类使用,其他类不能用某个类的内部类声明对象。
  \end{itemize}
\end{itemize}

\subsection{匿名类}
\subsubsection{和子类有关的匿名类}
\begin{itemize}
  \item \textbf{匿名类定义}:
  \begin{itemize}
    \item Java允许直接使用一个类的子类的类体创建一个子类对象。
    \item 创建子类对象时,使用父类的构造方法外加类体,称为匿名类。
    \item 例如:
    \begin{codeblock}
new Bank() {
    // 匿名类的类体
};
    \end{codeblock}
  \end{itemize}
\end{itemize}

\subsubsection{和接口有关的匿名类}
\begin{itemize}
  \item \textbf{接口匿名类}:
  \begin{itemize}
    \item Java允许直接用接口名和类体创建一个匿名对象,称为匿名类。
    \item 例如:
    \begin{codeblock}
new Computable() {
    // 实现接口的匿名类的类体
};
    \end{codeblock}
  \end{itemize}
\end{itemize}

\subsection{异常类}
\begin{itemize}
  \item \textbf{异常定义}:
  \begin{itemize}
    \item 程序运行时可能出现错误,异常处理改变程序控制流程,让程序对错误进行处理。
    \item 异常对象方法:
    \begin{enumerate}
      \item \texttt{getMessage()}
      \item \texttt{printStackTrace()}
      \item \texttt{toString()}
    \end{enumerate}
  \end{itemize}
\end{itemize}

\subsubsection{try-catch语句}
\begin{itemize}
  \item \textbf{异常处理语句}:
  \begin{itemize}
    \item Java使用\texttt{try-catch}语句处理异常,格式如下:
    \begin{codeblock}
try {
    // 可能发生异常的语句
} catch(ExceptionSubClass1 e) {
    // 异常处理
} catch(ExceptionSubClass2 e) {
    // 异常处理
}
    \end{codeblock}
  \end{itemize}
\end{itemize}

\subsubsection{自定义异常类}
\begin{itemize}
  \item \textbf{自定义异常}:
  \begin{itemize}
    \item 扩展\texttt{Exception}类定义自己的异常类。
    \item 方法在声明时使用\texttt{throws}关键字声明要产生的异常,在方法体中用\texttt{throw}抛出异常对象。
  \end{itemize}
\end{itemize}

\subsubsection{finally子语句}
\begin{itemize}
  \item \textbf{finally语句}:
  \begin{itemize}
    \item \texttt{try-catch}语句可以带\texttt{finally}子语句,格式如下:
    \begin{codeblock}
try {
    // 可能发生异常的语句
} catch(ExceptionSubClass e) {
    // 异常处理
} finally {
    // 无论是否发生异常都执行
}
    \end{codeblock}
    \item 执行机制:无论\texttt{try}部分是否发生异常,\texttt{finally}子语句都会执行。
    \item 特殊情况:\texttt{try-catch}中执行\texttt{return}语句,\texttt{finally}仍然执行;执行\texttt{System.exit(0)}不执行\texttt{finally}。
  \end{itemize}
\end{itemize}

\subsection{断言}
\begin{itemize}
  \item \textbf{断言语句}:
  \begin{itemize}
    \item 用于调试代码阶段,发现致命错误。
    \item 使用\texttt{assert}关键字声明断言语句,有两种格式:
    \begin{codeblock}
assert booleanExpression;
assert booleanExpression: messageException;
    \end{codeblock}
  \end{itemize}
\end{itemize}

\section{常用实用类}

\subsection{导读}
\begin{itemize}
  \item 主要内容:
  \begin{itemize}
    \item String类
    \item StringBuffer类
    \item StringTokenizer类
    \item Date类
    \item Calendar类
    \item Math与BigInteger类
    \item DecimalFormat类
    \item Pattern与Matcher类
    \item Scanner类
  \end{itemize}
\end{itemize}

\subsection{String类}
\begin{itemize}
  \item \textbf{String类概述}:
  \begin{itemize}
    \item 位于\texttt{java.lang}包中,用于创建字符串变量,字符串变量是对象。
  \end{itemize}
\end{itemize}

\subsubsection{构造字符串对象}
\begin{enumerate}
  \item \textbf{常量对象}:
  \begin{itemize}
    \item 用双引号括起的字符序列,如:"你好"、"12.97"、"boy"等。
  \end{itemize}
  \item \textbf{字符串对象}:
  \begin{itemize}
    \item 声明:\texttt{String s;}
    \item 构造方法:
    \begin{codeblock}
String(s)
String(char a[])
String(char a[], int startIndex, int count)
    \end{codeblock}
  \end{itemize}
  \item \textbf{引用字符串常量对象}:
  \begin{codeblock}
String s1 = "How are you";
String s2 = "How are you";
  \end{codeblock}
\end{enumerate}

\subsubsection{String 类的常用方法}
\begin{enumerate}
  \item \textbf{获取长度}:\texttt{public int length()}
  \item \textbf{比较字符串}:\texttt{public boolean equals(String s)}
  \item \textbf{前缀、后缀判断}:
  \begin{itemize}
    \item \texttt{public boolean startsWith(String s)}
    \item \texttt{public boolean endsWith(String s)}
  \end{itemize}
  \item \textbf{按字典序比较}:
  \begin{itemize}
    \item \texttt{public int compareTo(String s)}
    \item \texttt{public int compareToIgnoreCase(String s)}
  \end{itemize}
  \item \textbf{判断是否包含子字符串}:\texttt{public boolean contains(String s)}
  \item \textbf{检索子字符串位置}:
  \begin{itemize}
    \item \texttt{public int indexOf(String s)}
    \item \texttt{public int indexOf(String s, int startpoint)}
    \item \texttt{public int lastIndexOf(String s)}
  \end{itemize}
  \item \textbf{获取子字符串}:
  \begin{itemize}
    \item \texttt{public String substring(int startpoint)}
    \item \texttt{public String substring(int start, int end)}
  \end{itemize}
  \item \textbf{去除前后空格}:\texttt{public String trim()}
\end{enumerate}

\subsubsection{字符串与基本数据的相互转化}
\begin{itemize}
  \item \textbf{字符串转化为基本数据类型}:
  \begin{itemize}
    \item \texttt{Integer.parseInt(String s)}将字符串转化为int型数据
    \item 类似方法存在于\texttt{Byte}、\texttt{Short}、\texttt{Long}、\texttt{Float}、\texttt{Double}类中。
  \end{itemize}
  \item \textbf{基本数据类型转化为字符串}:
  \begin{itemize}
    \item \texttt{String.valueOf(byte n)}等方法。
    \item \texttt{Long.toBinaryString(long i)}等方法得到整数的各种进制字符串表示。
  \end{itemize}
\end{itemize}

\subsubsection{对象的字符串表示}
\begin{itemize}
  \item \textbf{toString()方法}:
  \begin{itemize}
    \item \texttt{Object}类中的\texttt{public String toString()}方法返回对象的字符串表示。
  \end{itemize}
\end{itemize}

\subsubsection{字符串与字符、字节数组}
\begin{enumerate}
  \item \textbf{字符串与字符数组}:
  \begin{itemize}
    \item 构造方法:\texttt{String(char[])}和\texttt{String(char[], int offset, int length)}
    \item 将字符串存放到字符数组中:\texttt{public void getChars(int start, int end, char c[], int offset)}
    \item 将字符串中的全部字符存放在一个字符数组中:\texttt{public char[] toCharArray()}
  \end{itemize}
  \item \textbf{字符串与字节数组}:
  \begin{itemize}
    \item 构造方法:\texttt{String(byte[])}和\texttt{String(byte[], int offset, int length)}
    \item 将字符串转化为字节数组:\texttt{public byte[] getBytes()}
    \item 使用指定字符编码转化为字节数组:\texttt{public byte[] getBytes(String charsetName)}
  \end{itemize}
\end{enumerate}

\subsubsection{正则表达式及字符串的替换与分解}
\begin{enumerate}
  \item \textbf{正则表达式}:
  \begin{itemize}
    \item 使用\texttt{public boolean matches(String regex)}方法判断当前字符串是否与指定正则表达式匹配。
  \end{itemize}
  \item \textbf{字符串的替换}:
  \begin{itemize}
    \item 使用\texttt{public String replaceAll(String regex, String replacement)}方法将匹配正则表达式的子字符串替换。
  \end{itemize}
  \item \textbf{字符串的分解}:
  \begin{itemize}
    \item 使用\texttt{public String[] split(String regex)}方法分解字符串。
  \end{itemize}
\end{enumerate}

\subsection{StringBuffer类}
\subsubsection{StringBuffer对象的创建}
\begin{itemize}
  \item \textbf{构造方法}:
  \begin{itemize}
    \item \texttt{StringBuffer()}
    \item \texttt{StringBuffer(int size)}
    \item \texttt{StringBuffer(String s)}
  \end{itemize}
\end{itemize}

\subsubsection{StringBuffer类的常用方法}
\begin{enumerate}
  \item \textbf{追加字符串}:\texttt{StringBuffer append(String s)}
  \item \textbf{获取单个字符}:\texttt{public char charAt(int n)}
  \item \textbf{替换字符}:\texttt{public void setCharAt(int n, char ch)}
  \item \textbf{插入字符串}:\texttt{StringBuffer insert(int index, String str)}
  \item \textbf{翻转字符串}:\texttt{public StringBuffer reverse()}
  \item \textbf{删除子字符串}:\texttt{StringBuffer delete(int startIndex, int endIndex)}
  \item \textbf{替换子字符串}:\texttt{StringBuffer replace(int startIndex, int endIndex, String str)}
\end{enumerate}

\subsection{StringTokenizer类}
\begin{itemize}
  \item \textbf{构造方法}:
  \begin{itemize}
    \item \texttt{StringTokenizer(String s)}
    \item \texttt{StringTokenizer(String s, String delim)}
  \end{itemize}
  \item \textbf{方法}:
  \begin{enumerate}
    \item \texttt{nextToken()}
    \item \texttt{hasMoreTokens()}
    \item \texttt{countTokens()}
  \end{enumerate}
\end{itemize}

\subsection{Date类}
\subsubsection{构造Date对象}
\begin{enumerate}
  \item \textbf{无参数构造方法}:
  \begin{codeblock}
Date nowTime = new Date();
  \end{codeblock}
  \item \textbf{有参数构造方法}:
  \begin{codeblock}
Date date1 = new Date(1000);
Date date2 = new Date(-1000);
  \end{codeblock}
\end{enumerate}

\subsubsection{日期格式化}
\begin{itemize}
  \item \textbf{使用SimpleDateFormat}:
  \begin{codeblock}
SimpleDateFormat format = new SimpleDateFormat("pattern");
format(Date date)
  \end{codeblock}
\end{itemize}

\subsection{Calendar类}
\begin{enumerate}
  \item \textbf{初始化日历对象}:
  \begin{codeblock}
Calendar calendar = Calendar.getInstance();
  \end{codeblock}
  \item \textbf{设置时间}:
  \begin{codeblock}
public final void set(int year, int month, int date)
public final void set(int year, int month, int date, int hour, int minute)
public final void set(int year, int month, int date, int hour, int minute, int second)
  \end{codeblock}
  \item \textbf{获取时间信息}:
  \begin{codeblock}
public int get(int field)
  \end{codeblock}
  \item \textbf{获取毫秒表示}:
  \begin{codeblock}
public long getTimeInMillis()
  \end{codeblock}
\end{enumerate}

\subsection{Math和BigInteger类}
\subsubsection{Math类}
\begin{itemize}
  \item \textbf{常用类方法}:
  \begin{enumerate}
    \item \texttt{public static long abs(double a)}
    \item \texttt{public static double max(double a, double b)}
    \item \texttt{public static double min(double a, double b)}
    \item \texttt{public static double random()}
    \item \texttt{public static double pow(double a, double b)}
    \item \texttt{public static double sqrt(double a)}
    \item \texttt{public static double log(double a)}
    \item \texttt{public static double sin(double a)}
    \item \texttt{public static double asin(double a)}
  \end{enumerate}
\end{itemize}

\subsubsection{BigInteger类}
\begin{itemize}
  \item \textbf{构造方法}:\texttt{public BigInteger(String val)}
  \item \textbf{常用类方法}:
  \begin{enumerate}
    \item \texttt{public BigInteger add(BigInteger val)}
    \item \texttt{public BigInteger subtract(BigInteger val)}
    \item \texttt{public BigInteger multiply(BigInteger val)}
    \item \texttt{public BigInteger divide(BigInteger val)}
    \item \texttt{public BigInteger remainder(BigInteger val)}
    \item \texttt{public int compareTo(BigInteger val)}
    \item \texttt{public BigInteger abs()}
    \item \texttt{public BigInteger pow(int a)}
    \item \texttt{public String toString()}
    \item \texttt{public String toString(int p)}
  \end{enumerate}
\end{itemize}

\subsection{DecimalFormat类}
\subsubsection{格式化数字}
\begin{itemize}
  \item \textbf{格式化整数位和小数位}:\texttt{DecimalFormat format = new DecimalFormat("pattern");}
  \item \textbf{格式化为百分数或千分数}:在模式尾加\texttt{\%}或\texttt{\textbackslash u2030}
  \item \textbf{格式化为科学计数}:在模式尾加\texttt{E0}
  \item \textbf{格式化为货币值}:在模式尾加货币符号(如\texttt{\$}、\texttt{\textbackslash ¥})
\end{itemize}

\subsubsection{将格式化字符串转化为数字}
\begin{itemize}
  \item \textbf{将格式化字符串转化为数字}:\texttt{DecimalFormat df = new DecimalFormat("pattern");}
  \item \textbf{解析字符串}:\texttt{Number num = df.parse("string");}
  \item \textbf{获取数值}:\texttt{double d = num.doubleValue();}
\end{itemize}

\subsection{Pattern与Matcher类}
\subsubsection{模式对象}
\begin{itemize}
  \item \textbf{创建模式对象}:\texttt{Pattern p = Pattern.compile("regex");}
  \item \textbf{使用flags创建模式对象}:\texttt{Pattern.compile(String regex, int flags)}
\end{itemize}

\subsubsection{匹配对象}
\begin{itemize}
  \item \textbf{创建匹配对象}:\texttt{Matcher m = p.matcher(CharSequence input);}
  \item \textbf{常用方法}:
  \begin{enumerate}
    \item \texttt{public boolean find()}
    \item \texttt{public boolean matches()}
    \item \texttt{public boolean lookingAt()}
    \item \texttt{public boolean find(int start)}
    \item \texttt{public String replaceAll(String replacement)}
    \item \texttt{public String replaceFirst(String replacement)}
  \end{enumerate}
\end{itemize}

\subsection{Scanner类}
\begin{enumerate}
  \item \textbf{默认分隔标记解析字符串}:
  \begin{codeblock}
Scanner scanner = new Scanner("string");
  \end{codeblock}
  \item \textbf{使用正则表达式作为分隔标记解析字符串}:
  \begin{codeblock}
Scanner scanner = new Scanner("string").useDelimiter("regex");
  \end{codeblock}
\end{enumerate}

\section{输入流与输出流}

\subsection{导读}
\begin{itemize}
  \item 主要内容:
  \begin{itemize}
    \item 字节流与字符流
    \item 缓冲流
    \item 随机流
    \item 数组流
    \item 数据流
    \item 对象流
    \item 序列化与对象克隆
    \item 文件锁
    \item 使用Scanner解析文件
  \end{itemize}
\end{itemize}

\subsection{概述}
\begin{itemize}
  \item \textbf{输入、输出流}:
  \begin{itemize}
    \item 提供一条通道用于读取源数据或将数据传送到目的地。
    \item 输入流指向源,程序从输入流中读取数据。
    \item 输出流指向目的地,程序通过输出流写入数据。
  \end{itemize}
\end{itemize}

\subsection{File类}
\subsubsection{File对象}
\begin{itemize}
  \item \textbf{用于获取文件信息,不涉及读写操作}。
  \item 构造方法:
  \begin{codeblock}
File(String filename);
File(String directoryPath, String filename);
File(File f, String filename);
  \end{codeblock}
\end{itemize}

\subsubsection{文件的属性}
\begin{itemize}
  \item \textbf{获取文件信息的方法}:
  \begin{enumerate}
    \item \texttt{getName()}: 获取文件名
    \item \texttt{canRead()}: 判断文件是否可读
    \item \texttt{canWrite()}: 判断文件是否可写
    \item \texttt{exists()}: 判断文件是否存在
    \item \texttt{length()}: 获取文件长度(字节)
    \item \texttt{getAbsolutePath()}: 获取绝对路径
    \item \texttt{getParent()}: 获取父目录
    \item \texttt{isFile()}: 判断是否为文件
    \item \texttt{isDirectory()}: 判断是否为目录
    \item \texttt{isHidden()}: 判断是否为隐藏文件
    \item \texttt{lastModified()}: 获取最后修改时间
  \end{enumerate}
\end{itemize}

\subsubsection{目录}
\begin{itemize}
  \item \textbf{创建目录}:
  \begin{codeblock}
mkdir(): 创建目录
  \end{codeblock}
  \item \textbf{列出目录中的文件}:
  \begin{codeblock}
list(): 以字符串形式返回目录下所有文件
listFiles(): 以File对象形式返回目录下所有文件
list(FilenameFilter obj): 返回目录下指定类型文件
listFiles(FilenameFilter obj): 返回目录下指定类型文件
  \end{codeblock}
\end{itemize}

\subsubsection{文件的创建与删除}
\begin{itemize}
  \item \textbf{创建文件}:
  \begin{codeblock}
createNewFile(): 创建新文件
  \end{codeblock}
  \item \textbf{删除文件}:
  \begin{codeblock}
delete(): 删除文件
  \end{codeblock}
\end{itemize}

\subsubsection{运行可执行文件}
\begin{itemize}
  \item \textbf{使用Runtime类}:
  \begin{codeblock}
Runtime.getRuntime().exec(String command): 打开可执行文件或执行操作
  \end{codeblock}
\end{itemize}

\subsection{字节流与字符流}
\begin{itemize}
  \item \textbf{字节流}:
  \begin{itemize}
    \item \texttt{InputStream}类:字节输入流的父类
    \item \texttt{OutputStream}类:字节输出流的父类
  \end{itemize}
\end{itemize}

\subsubsection{InputStream类与OutputStream类}
\begin{itemize}
  \item \textbf{InputStream类常用方法}:
  \begin{codeblock}
read(): 读取字节
read(byte b[]): 读取字节数组
read(byte b[], int off, int len): 读取指定长度字节数组
close(): 关闭流
skip(long numBytes): 跳过字节
  \end{codeblock}
  \item \textbf{OutputStream类常用方法}:
  \begin{codeblock}
write(int n): 写入字节
write(byte b[]): 写入字节数组
write(byte b[], int off, int len): 写入指定长度字节数组
close(): 关闭流
  \end{codeblock}
\end{itemize}

\subsubsection{Reader类与Writer类}
\begin{itemize}
  \item \textbf{Reader类常用方法}:
  \begin{codeblock}
read(): 读取字符
read(char b[]): 读取字符数组
read(char b[], int off, int len): 读取指定长度字符数组
close(): 关闭流
skip(long numBytes): 跳过字符
  \end{codeblock}
  \item \textbf{Writer类常用方法}:
  \begin{codeblock}
write(int n): 写入字符
write(char b[]): 写入字符数组
write(char b[], int off, int len): 写入指定长度字符数组
close(): 关闭流
  \end{codeblock}
\end{itemize}

\subsubsection{关闭流}
\begin{itemize}
  \item \textbf{关闭流}:
  \begin{itemize}
    \item 调用\texttt{close()}方法显式关闭流,保证缓冲区内容写到目的地。
  \end{itemize}
\end{itemize}

\subsection{文件字节流}
\subsubsection{文件字节输入流}
\begin{itemize}
  \item \textbf{创建文件字节输入流}:
  \begin{codeblock}
FileInputStream(String name)
FileInputStream(File file)
  \end{codeblock}
  \item \textbf{读取方法}:
  \begin{codeblock}
int read()
int read(byte b[])
int read(byte b[], int off, int len)
  \end{codeblock}
\end{itemize}

\subsubsection{文件字节输出流}
\begin{itemize}
  \item \textbf{创建文件字节输出流}:
  \begin{codeblock}
FileOutputStream(String name)
FileOutputStream(File file)
  \end{codeblock}
  \item \textbf{写入方法}:
  \begin{codeblock}
void write(byte b[])
void write(byte b[], int off, int len)
  \end{codeblock}
\end{itemize}

\subsection{文件字符流}
\begin{itemize}
  \item \textbf{文件字符输入流和输出流}:
  \begin{codeblock}
FileReader(String filename)
FileReader(File filename)
FileWriter(String filename)
FileWriter(File filename)
  \end{codeblock}
\end{itemize}

\subsection{缓冲流}
\begin{itemize}
  \item \textbf{BufferedReader和BufferedWriter}:
  \begin{itemize}
    \item 构造方法:
    \begin{codeblock}
BufferedReader(Reader in)
BufferedWriter(Writer out)
    \end{codeblock}
    \item 方法:
    \begin{codeblock}
readLine()
write(String s, int off, int len)
newLine()
    \end{codeblock}
  \end{itemize}
\end{itemize}

\subsection{随机流}
\begin{itemize}
  \item \textbf{RandomAccessFile类}:
  \begin{itemize}
    \item 构造方法:
    \begin{codeblock}
RandomAccessFile(String name, String mode)
RandomAccessFile(File file, String mode)
    \end{codeblock}
    \item 方法:
    \begin{codeblock}
seek(long a)
getFilePointer()
    \end{codeblock}
  \end{itemize}
\end{itemize}

\subsection{数组流}
\subsubsection{字节数组流}
\begin{itemize}
  \item \textbf{ByteArrayInputStream和ByteArrayOutputStream}:
  \begin{itemize}
    \item 构造方法:
    \begin{codeblock}
ByteArrayInputStream(byte[] buf)
ByteArrayInputStream(byte[] buf, int offset, int length)
ByteArrayOutputStream()
ByteArrayOutputStream(int size)
    \end{codeblock}
    \item 方法:
    \begin{codeblock}
read()
read(byte[] b, int off, int len)
write(int b)
write(byte[] b, int off, int len)
toByteArray()
    \end{codeblock}
  \end{itemize}
\end{itemize}

\subsubsection{字符数组流}
\begin{itemize}
  \item \textbf{CharArrayReader和CharArrayWriter}
\end{itemize}

\subsection{数据流}
\begin{itemize}
  \item \textbf{DataInputStream和DataOutputStream}:
  \begin{itemize}
    \item 构造方法:
    \begin{codeblock}
DataInputStream(InputStream in)
DataOutputStream(OutputStream out)
    \end{codeblock}
  \end{itemize}
\end{itemize}

\subsection{对象流}
\begin{itemize}
  \item \textbf{ObjectInputStream和ObjectOutputStream}:
  \begin{itemize}
    \item 构造方法:
    \begin{codeblock}
ObjectInputStream(InputStream in)
ObjectOutputStream(OutputStream out)
    \end{codeblock}
    \item 方法:
    \begin{codeblock}
writeObject(Object obj)
readObject()
    \end{codeblock}
  \end{itemize}
\end{itemize}

\subsection{序列化与对象克隆}
\begin{itemize}
  \item \textbf{序列化}:
  \begin{itemize}
    \item 实现\texttt{Serializable}接口
  \end{itemize}
  \item \textbf{对象克隆}:
  \begin{itemize}
    \item 调用\texttt{clone()}方法
  \end{itemize}
\end{itemize}

\subsection{文件锁}
\begin{itemize}
  \item \textbf{文件锁功能}:
  \begin{itemize}
    \item 使用FileLock、FileChannel类
    \item 步骤:
    \begin{enumerate}
      \item 使用RandomAccessFile流建立指向文件的流对象
      \item 调用\texttt{getChannel()}方法获得FileChannel对象
      \item 调用\texttt{tryLock()}或\texttt{lock()}方法获得FileLock对象
    \end{enumerate}
  \end{itemize}
\end{itemize}

\subsection{使用Scanner解析文件}
\begin{enumerate}
  \item \textbf{默认分隔标记解析文件}:
  \begin{codeblock}
Scanner sc = new Scanner(new File("filename"));
  \end{codeblock}
  \item \textbf{使用正则表达式作为分隔标记解析文件}:
  \begin{codeblock}
Scanner sc = new Scanner(new File("filename")).useDelimiter("regex");
  \end{codeblock}
\end{enumerate}

\subsection{总结}
\begin{itemize}
  \item 本章详细介绍了Java中的输入流与输出流,包括文件操作、字节流与字符流、缓冲流、随机流、数组流、数据流、对象流、序列化与对象克隆、文件锁以及使用Scanner解析文件。这些内容是Java I/O编程的基础,掌握这些知识可以有效地进行文件和数据的读写操作。
\end{itemize}

\section{组件及事件处理}

\subsection{导读}
\begin{itemize}
  \item 主要内容:
  \begin{itemize}
    \item Java Swing概述
    \item 窗口
    \item 常用组件与布局
    \item 处理事件
    \item 使用MVC结构
    \item 对话框
    \item 发布GUI程序
  \end{itemize}
\end{itemize}

\subsection{Java Swing概述}
\begin{itemize}
  \item \textbf{Java的java.awt包}:
  \begin{itemize}
    \item 提供了许多用来设计GUI的组件类。
  \end{itemize}
\end{itemize}

\subsection{窗口}
\subsubsection{JFrame类}
\begin{itemize}
  \item \textbf{JFrame类的实例是一个底层容器,其他组件必须被添加到底层容器中,以便借助这个底层容器和操作系统进行信息交互}。
  \item JFrame类是Container类的间接子类。
\end{itemize}

\subsubsection{JFrame常用方法}
\begin{itemize}
  \item \textbf{常用方法}:
  \begin{codeblock}
JFrame(): 创建一个无标题的窗口。
JFrame(String s): 创建标题为s的窗口。
setBounds(int a, int b, int width, int height): 设置窗口的初始位置和大小。
setSize(int width, int height): 设置窗口的大小。
setLocation(int x, int y): 设置窗口的位置。
setVisible(boolean b): 设置窗口是否可见。
setResizable(boolean b): 设置窗口是否可调整大小。
dispose(): 撤消当前窗口,并释放当前窗口所使用的资源。
setExtendedState(int state): 设置窗口的扩展状态。
setDefaultCloseOperation(int operation): 设置窗口关闭操作。
  \end{codeblock}
\end{itemize}

\subsubsection{菜单条、菜单、菜单项}
\begin{itemize}
  \item \textbf{菜单条}:
  \begin{codeblock}
JMenuBar: 创建菜单条。
setJMenuBar(JMenuBar bar): 将菜单条添加到窗口中。
  \end{codeblock}
  \item \textbf{菜单}:
  \begin{codeblock}
JMenu: 创建菜单。
add(JMenuItem item): 向菜单增加菜单项。
  \end{codeblock}
  \item \textbf{菜单项}:
  \begin{codeblock}
JMenuItem: 创建菜单项。
setEnabled(boolean b): 设置菜单项是否可被选择。
setAccelerator(KeyStroke keyStroke): 为菜单项设置快捷键。
  \end{codeblock}
  \item \textbf{嵌入子菜单}:
  \begin{codeblock}
JMenu是JMenuItem的子类,可以嵌入子菜单。
  \end{codeblock}
  \item \textbf{菜单上的图标}:
  \begin{codeblock}
使用Icon和ImageIcon类创建图标。
  \end{codeblock}
\end{itemize}

\subsection{常用组件与布局}
\subsubsection{常用组件}
\begin{itemize}
  \item \textbf{常用组件}:
  \begin{codeblock}
JTextField: 创建文本框。
JTextArea: 创建文本区。
JButton: 创建按钮。
JLabel: 创建标签。
JCheckBox: 创建选择框。
JRadioButton: 创建单选按钮。
JComboBox: 创建下拉列表。
JPasswordField: 创建密码框。
  \end{codeblock}
\end{itemize}

\subsubsection{常用容器}
\begin{itemize}
  \item \textbf{常用容器}:
  \begin{codeblock}
JPanel: 创建面板,默认布局是FlowLayout。
JScrollPane: 创建滚动窗格。
JSplitPane: 创建拆分窗格。
JLayeredPane: 创建分层窗格。
  \end{codeblock}
\end{itemize}

\subsubsection{常用布局}
\begin{itemize}
  \item \textbf{布局管理器}:
  \begin{codeblock}
FlowLayout: 顺序布局。
BorderLayout: 边界布局。
CardLayout: 卡片布局。
GridLayout: 网格布局。
null布局: 空布局,精确定位组件的位置和大小。
  \end{codeblock}
\end{itemize}

\subsubsection{选项卡窗格}
\begin{itemize}
  \item \textbf{JTabbedPane}:
  \begin{codeblock}
创建选项卡窗格,默认布局是CardLayout。
  \end{codeblock}
\end{itemize}

\subsection{处理事件}
\subsubsection{事件处理模式}
\begin{itemize}
  \item \textbf{事件源}:
  \begin{codeblock}
能够产生事件的对象。
  \end{codeblock}
  \item \textbf{监视器}:
  \begin{codeblock}
事件源通过方法将对象注册为监视器。
  \end{codeblock}
  \item \textbf{处理事件的接口}:
  \begin{codeblock}
监视器对象自动调用方法处理事件。
  \end{codeblock}
\end{itemize}

\subsubsection{ActionEvent事件}
\begin{itemize}
  \item \textbf{事件源}:
  \begin{codeblock}
文本框、按钮、菜单项、密码框和单选按钮。
  \end{codeblock}
  \item \textbf{注册监视器}:
  \begin{codeblock}
addActionListener(ActionListener listener): 注册监视器。
  \end{codeblock}
  \item \textbf{ActionListener接口}:
  \begin{codeblock}
actionPerformed(ActionEvent e): 处理事件。
  \end{codeblock}
  \item \textbf{ActionEvent类方法}:
  \begin{codeblock}
getSource(): 获取事件源对象。
getActionCommand(): 获取命令字符串。
  \end{codeblock}
\end{itemize}

\subsubsection{ItemEvent事件}
\begin{itemize}
  \item \textbf{事件源}:
  \begin{codeblock}
选择框、下拉列表。
  \end{codeblock}
  \item \textbf{注册监视器}:
  \begin{codeblock}
addItemListener(ItemListener listener): 注册监视器。
  \end{codeblock}
  \item \textbf{ItemListener接口}:
  \begin{codeblock}
itemStateChanged(ItemEvent e): 处理事件。
  \end{codeblock}
  \item \textbf{ItemEvent类方法}:
  \begin{codeblock}
getItemSelectable(): 返回事件源。
  \end{codeblock}
\end{itemize}

\subsubsection{DocumentEvent事件}
\begin{itemize}
  \item \textbf{事件源}:
  \begin{codeblock}
文本区维护的文档。
  \end{codeblock}
  \item \textbf{注册监视器}:
  \begin{codeblock}
addDocumentListener(DocumentListener listener): 注册监视器。
  \end{codeblock}
  \item \textbf{DocumentListener接口}:
  \begin{codeblock}
changedUpdate(DocumentEvent e)
removeUpdate(DocumentEvent e)
insertUpdate(DocumentEvent e)
  \end{codeblock}
\end{itemize}

\subsubsection{MouseEvent事件}
\begin{itemize}
  \item \textbf{事件源}:
  \begin{codeblock}
组件上的鼠标事件。
  \end{codeblock}
  \item \textbf{注册监视器}:
  \begin{codeblock}
addMouseListener(MouseListener listener)
addMouseMotionListener(MouseMotionListener listener)
  \end{codeblock}
  \item \textbf{MouseListener接口}:
  \begin{codeblock}
mousePressed(MouseEvent)
mouseReleased(MouseEvent)
mouseEntered(MouseEvent)
mouseExited(MouseEvent)
mouseClicked(MouseEvent)
  \end{codeblock}
  \item \textbf{MouseMotionListener接口}:
  \begin{codeblock}
mouseDragged(MouseEvent)
mouseMoved(MouseEvent)
  \end{codeblock}
\end{itemize}

\subsubsection{焦点事件}
\begin{itemize}
  \item \textbf{事件源}:
  \begin{codeblock}
组件可以触发焦点事件。
  \end{codeblock}
  \item \textbf{注册监视器}:
  \begin{codeblock}
addFocusListener(FocusListener listener)
  \end{codeblock}
  \item \textbf{FocusListener接口}:
  \begin{codeblock}
focusGained(FocusEvent e)
focusLost(FocusEvent e)
  \end{codeblock}
\end{itemize}

\subsubsection{键盘事件}
\begin{itemize}
  \item \textbf{事件源}:
  \begin{codeblock}
组件处于激活状态时触发键盘事件。
  \end{codeblock}
  \item \textbf{注册监视器}:
  \begin{codeblock}
addKeyListener(KeyListener listener)
  \end{codeblock}
  \item \textbf{KeyListener接口}:
  \begin{codeblock}
keyPressed(KeyEvent e)
keyTyped(KeyEvent e)
keyReleased(KeyEvent e)
  \end{codeblock}
\end{itemize}

\subsubsection{匿名类实例或窗口做监视器}
\begin{itemize}
  \item \textbf{匿名类}:
  \begin{codeblock}
匿名类的外嵌类的成员变量在匿名类中有效。
  \end{codeblock}
  \item \textbf{窗口做监视器}:
  \begin{codeblock}
事件源所在类的实例作为监视器。
  \end{codeblock}
\end{itemize}

\subsubsection{事件总结}
\begin{itemize}
  \item \textbf{授权模式}:
  \begin{codeblock}
基于授权模式的事件处理。
  \end{codeblock}
  \item \textbf{接口回调}:
  \begin{codeblock}
事件源发生事件时,接口回调方法。
  \end{codeblock}
  \item \textbf{方法绑定}:
  \begin{codeblock}
事件源触发事件后,调用相应的方法。
  \end{codeblock}
  \item \textbf{保持松耦合}:
  \begin{codeblock}
事件处理保持系统的松耦合性。
  \end{codeblock}
\end{itemize}

\subsection{使用MVC结构}
\begin{itemize}
  \item \textbf{MVC结构}:
  \begin{codeblock}
模型(Model):存储数据的对象。
视图(View):显示数据的对象。
控制器(Controller):处理用户交互的对象。
  \end{codeblock}
\end{itemize}

\subsection{对话框}
\subsubsection{对话框类型}
\begin{itemize}
  \item \textbf{对话框类型}:
  \begin{codeblock}
无模式对话框:能再激活其它窗口,不堵塞线程执行。
有模式对话框:只响应对话框内部的事件,堵塞其它线程执行。
  \end{codeblock}
\end{itemize}

\subsubsection{消息对话框}
\begin{itemize}
  \item \textbf{JOptionPane类}:
  \begin{codeblock}
showMessageDialog(Component parentComponent, String message, String title, int messageType): 创建消息对话框。
  \end{codeblock}
\end{itemize}

\subsubsection{输入对话框}
\begin{itemize}
  \item \textbf{JOptionPane类}:
  \begin{codeblock}
showInputDialog(Component parentComponent, Object message, String title, int messageType): 创建输入对话框。
  \end{codeblock}
\end{itemize}

\subsubsection{确认对话框}
\begin{itemize}
  \item \textbf{JOptionPane类}:
  \begin{codeblock}
showConfirmDialog(Component parentComponent, Object message, String title, int optionType): 创建确认对话框。
  \end{codeblock}
\end{itemize}

\subsubsection{颜色对话框}
\begin{itemize}
  \item \textbf{JColorChooser类}:
  \begin{codeblock}
showDialog(Component component, String title, Color initialColor): 创建颜色对话框。
  \end{codeblock}
\end{itemize}

\subsubsection{文件对话框}
\begin{itemize}
  \item \textbf{JFileChooser类}:
  \begin{codeblock}
showSaveDialog(Component a): 显示保存文件对话框。
showOpenDialog(Component a): 显示打开文件对话框。
  \end{codeblock}
\end{itemize}

\subsubsection{自定义对话框}
\begin{itemize}
  \item \textbf{JDialog类}:
  \begin{codeblock}
创建对话框,通过建立JDialog的子类来实现。
  \end{codeblock}
\end{itemize}

\subsection{发布GUI程序}
\begin{itemize}
  \item \textbf{发布步骤}:
  \begin{enumerate}
    \item 编写清单文件。
    \item 生成JAR文件。
    \item 在安装了Java运行环境的计算机上运行JAR文件。
  \end{enumerate}
\end{itemize}

\subsection{总结}
\begin{itemize}
  \item 本章详细介绍了Java中的组件及事件处理,包括Java Swing的基本概念、窗口、常用组件与布局、事件处理模式、使用MVC结构、对话框和发布GUI程序。这些内容是Java GUI编程的重要组成部分,掌握这些知识可以有效地开发和发布Java图形用户界面程序。
\end{itemize}

\section*{补充内容:泛型与集合框架}

\subsection{导读}
\begin{itemize}
  \item 主要内容:
  \begin{itemize}
    \item 泛型
    \item 集合框架
  \end{itemize}
\end{itemize}

\subsection{泛型}
\subsubsection{泛型的定义}
\begin{itemize}
  \item 泛型实现了参数化类型的能力,使代码可以应用于多种类型。
  \item 目的:使类、方法和接口具备广泛的表达能力。
  \item Java允许定义泛型类、泛型接口和泛型方法。
\end{itemize}

\subsubsection{泛型类}
\begin{itemize}
  \item 定义泛型类的格式:
  \begin{codeblock}
public class 类名<类型参数表> {
    // 成员变量
    // 成员方法
}
  \end{codeblock}
  \item 例子:定义一个Pair类,其中有两个私有成员变量\texttt{first}和\texttt{second},类型可以为任意类型。
\end{itemize}

\subsubsection{泛型方法}
\begin{itemize}
  \item 定义泛型方法的格式:
  \begin{codeblock}
public <T> 返回类型 方法名(参数列表) {
    // 方法体
}
  \end{codeblock}
\end{itemize}

\subsubsection{泛型接口}
\begin{itemize}
  \item 定义泛型接口的格式:
  \begin{codeblock}
interface 接口名称<类型参数表> {
    // 方法原型
}
  \end{codeblock}
  \item 例子:定义Info接口,声明的\texttt{print()}方法可以输出任意类型对象的提示信息。
\end{itemize}

\subsection{集合框架}
\subsubsection{集合框架概述}
\begin{itemize}
  \item Java集合框架提供一系列能有效组织和操作数据的数据结构。
  \item 主要包含以下三部分:
  \begin{enumerate}
    \item 对外的接口
    \item 接口的实现
    \item 对集合进行运算的算法
  \end{enumerate}
\end{itemize}

\subsubsection{Iterator接口}
\begin{itemize}
  \item \textbf{Iterator接口}:用于遍历集合中的元素。
  \item 隐藏各种集合类的底层实现细节,提供统一的编程接口。
\end{itemize}

\subsubsection{常用集合类}
\begin{enumerate}
  \item \textbf{ArrayList类}:
  \begin{itemize}
    \item 动态数组,实现了List接口。
    \item 例子:创建ArrayList对象,添加元素并使用迭代器遍历集合。
  \end{itemize}
  \item \textbf{LinkedList类}:
  \begin{itemize}
    \item 双向链表,实现了List接口。
    \item 例子:创建两个链表,合并链表并删除元素。
  \end{itemize}
  \item \textbf{Stack类}:
  \begin{itemize}
    \item 栈,后进先出(LIFO)容器。
    \item 例子:把英文字符串按空格分隔后依次入栈,再依次出栈。
  \end{itemize}
  \item \textbf{HashMap类}:
  \begin{itemize}
    \item 映射表,存放键值对,通过键查找对应的值。
    \item 例子:使用HashMap类存放和查找键值对。
  \end{itemize}
\end{enumerate}

\section{Java多线程机制}

\subsection{导读}
\begin{itemize}
  \item 主要内容:
  \begin{itemize}
    \item 进程与线程
    \item Java中的线程
    \item 线程的创建
    \item 线程的生命周期
    \item 线程同步
    \item 线程联合
  \end{itemize}
\end{itemize}

\subsection{进程与线程}
\begin{itemize}
  \item \textbf{进程}:程序的一次执行过程,可以产生多个线程。
  \item \textbf{线程}:更小的执行单位,可以共享进程的内存。
\end{itemize}

\subsection{Java中的线程}
\begin{itemize}
  \item Java支持多线程,JVM管理线程调度。
  \item 线程有四种状态:新建、运行、中断和死亡。
\end{itemize}

\subsection{线程的创建}
\begin{itemize}
  \item 通过继承\texttt{Thread}类或实现\texttt{Runnable}接口创建线程。
  \item \texttt{Thread}类的方法:
  \begin{itemize}
    \item \texttt{start()}: 启动线程。
    \item \texttt{run()}: 线程的执行代码。
    \item \texttt{sleep(int millisecond)}: 使线程休眠。
  \end{itemize}
\end{itemize}

\subsection{线程的生命周期}
\begin{itemize}
  \item 线程在其生命周期中经历新建、就绪、运行、阻塞和死亡等状态。
\end{itemize}

\subsection{线程同步}
\begin{itemize}
  \item 使用\texttt{synchronized}关键字进行方法同步。
  \item 使用\texttt{wait()}, \texttt{notify()}, \texttt{notifyAll()}进行线程间通信。
\end{itemize}

\subsection{线程联合}
\begin{itemize}
  \item 使用\texttt{join()}方法使一个线程等待另一个线程执行完毕。
\end{itemize}

\section{Java网络编程}

\subsection{导读}
\begin{itemize}
  \item 主要内容:
  \begin{itemize}
    \item URL类
    \item InetAddress类
    \item 套接字
    \item UDP数据报
    \item 广播数据报
    \item Java远程调用(RMI)
  \end{itemize}
\end{itemize}

\subsection{URL类}
\begin{itemize}
  \item \textbf{URL类}:封装一个统一资源定位符(Uniform Resource Locator)。
  \item 通过\texttt{URL(String spec)}和\texttt{URL(String protocol, String host, String file)}构造URL对象。
  \item 使用\texttt{openStream()}方法获取输入流读取资源。
\end{itemize}

\subsection{InetAddress类}
\begin{itemize}
  \item 表示Internet上的主机地址,包含域名和IP地址。
  \item 使用\texttt{getByName(String host)}和\texttt{getLocalHost()}获取InetAddress对象。
\end{itemize}

\subsection{套接字}
\begin{itemize}
  \item 客户端使用\texttt{Socket}类与服务器建立连接。
  \item 服务器使用\texttt{ServerSocket}类监听客户端连接,使用\texttt{accept()}方法接收连接。
\end{itemize}

\subsection{UDP数据报}
\begin{itemize}
  \item 使用\texttt{DatagramPacket}和\texttt{DatagramSocket}类进行数据包的发送和接收。
  \item 通过\texttt{send(DatagramPacket p)}发送数据包,\texttt{receive(DatagramPacket p)}接收数据包。
\end{itemize}

\subsection{广播数据报}
\begin{itemize}
  \item 使用D类地址(224.0.0.0~224.255.255.255)进行数据报的广播。
\end{itemize}

\subsection{Java远程调用(RMI)}
\begin{itemize}
  \item 允许一个JVM调用另一个JVM中的对象方法。
  \item 远程对象实现\texttt{Remote}接口,使用\texttt{Naming}类绑定远程对象。
\end{itemize}

\section{JDBC数据库操作}

\subsection{导读}
\begin{itemize}
  \item 主要内容:
  \begin{itemize}
    \item Derby数据库
    \item 连接内置Derby数据库
    \item 操作表
    \item JDBC操作
    \item 使用预处理语句
  \end{itemize}
\end{itemize}

\subsection{Derby数据库}
\begin{itemize}
  \item JDK 1.6及以上版本提供的嵌入式数据库。
  \item 准备工作:将\texttt{derby.jar}、\texttt{derbynet.jar}和\texttt{derbyclient.jar}复制到JDK的\texttt{lib}目录中。
\end{itemize}

\subsection{连接内置Derby数据库}
\begin{itemize}
  \item 使用\texttt{connect 'jdbc:derby:数据库;create=true|false';}命令连接数据库。
\end{itemize}

\subsection{操作表}
\begin{itemize}
  \item \textbf{创建表}:
  \begin{codeblock}
create table 表名(字段1 属性, 字段2 属性, ...);
  \end{codeblock}
  \item \textbf{插入记录}:
  \begin{codeblock}
insert into 表名 values (字段1值, 字段2值, ...);
  \end{codeblock}
  \item \textbf{查询记录}:
  \begin{codeblock}
select * from 表名;
  \end{codeblock}
  \item \textbf{更新记录}:
  \begin{codeblock}
update 表名 set 字段名=新值 where 条件;
  \end{codeblock}
  \item \textbf{删除记录}:
  \begin{codeblock}
delete from 表名 where 条件;
  \end{codeblock}
\end{itemize}

\subsection{JDBC操作}
\begin{itemize}
  \item \textbf{加载驱动}:
  \begin{codeblock}
Class.forName("org.apache.derby.jdbc.EmbeddedDriver");
  \end{codeblock}
  \item \textbf{建立连接}:
  \begin{codeblock}
Connection conn = DriverManager.getConnection("jdbc:derby:数据库名");
  \end{codeblock}
  \item \textbf{创建语句}:
  \begin{codeblock}
Statement stmt = conn.createStatement();
  \end{codeblock}
  \item \textbf{执行查询}:
  \begin{codeblock}
ResultSet rs = stmt.executeQuery("SQL语句");
  \end{codeblock}
  \item \textbf{更新操作}:
  \begin{codeblock}
stmt.executeUpdate("SQL语句");
  \end{codeblock}
  \item \textbf{关闭连接}:
  \begin{codeblock}
conn.close();
  \end{codeblock}
\end{itemize}

\subsection{使用预处理语句}
\begin{itemize}
  \item \textbf{使用\texttt{PreparedStatement}}提高执行效率和安全性。
  \item \textbf{事务处理}:
  \begin{itemize}
    \item 使用\texttt{conn.setAutoCommit(false)}开启事务,\texttt{conn.commit()}提交事务,\texttt{conn.rollback()}回滚事务。
  \end{itemize}
\end{itemize}

\end{document}



\end{document}
